% LaTeX resume using res.cls
\documentclass[centered,overlapped]{res}
%\usepackage[margin=1.0in]{geometry}
%\topmargin=0in
\oddsidemargin -0.8in
\evensidemargin -0.8in
\textwidth=7.5in
%\itemsep=-0.75in
%\parsep=-0.5in

%\usepackage{helvetica} % uses helvetica postscript font (download helvetica.sty)
%\usepackage{newcent}   % uses new century schoolbook postscript font
\usepackage{xcolor}
\usepackage[colorlinks = true,
            linkcolor = blue,
            urlcolor  = blue,
            citecolor = blue,
            anchorcolor = blue]{hyperref}

\usepackage{enumitem}

\newlist{SubItemList}{itemize}{1}
\setlist[SubItemList]{label={$-$}}

\let\OldItem\item
\newcommand{\SubItemStart}[1]{%
\let\item\SubItemEnd
\begin{SubItemList}[resume]%
\OldItem #1%
}
\newcommand{\SubItemMiddle}[1]{%
\OldItem #1%
}
\newcommand{\SubItemEnd}[1]{%
\end{SubItemList}%
\let\item\OldItem
\item #1%
}
\newcommand*{\SubItem}[1]{%
\let\SubItem\SubItemMiddle%
\SubItemStart{#1}%
}%

\begin{document}

\name{Joana M. F. da Trindade}

\address{jmf@csail.mit.edu, \url{http://joana.fyi}}

\begin{resume}

\section{RESEARCH \\ INTERESTS} \textbf{temporal graphs, graph data management, query optimization, data processing systems}

\section{EDUCATION} \textbf{Massachusetts Institute of Technology (MIT)} \hfill Fall 2016-- present \\
  {\sl PhD Candidate in EECS,} CGPA 5.0/5.0 \\
  Thesis work (in-progress) on large-scale temporal graph data analytics.\\
  PIs: Prof. Sam Madden (Data Systems Group), and Prof. Julian Shun (Theory of Computation).
  \begin{itemize}  \itemsep -2pt
  \item Graduate EECS: Database Systems, Distributed Systems, Advances in Computer Vision, Graph Analytics (audited), Introduction to the Theory of Computation
  \item Minor: Fund. of Music Theory, Digital Music Processing
  \end{itemize}

  \textbf{University of Illinois at Urbana-Champaign (UIUC)}, {\sl M.S. in CS} \\ %\hfill 2009--2011 \\
  %{\sl Master of Science in Computer Science,}  GPA 3.88/4.00 \\
  Advisor: Prof. Marianne Winslett
  \begin{itemize}  \itemsep -2pt
  \item Thesis: Supporting Dynamic Queries and Annotations Over Data Graphs
  \item Graduate Coursework: Advanced Database Systems, Advanced Operating Systems, Algorithms, Cloud Computing Infrastructure, Fault-Tolerant Digital Systems Design, Parallel Computer Architecture, Secure Data Management
  \end{itemize}

  \textbf{Universidade Federal do Rio Grande do Sul and TU Kaiserslautern}, {\sl B.S. in CS} \\ %\hfill 2003--2008 \\
 % {\sl Bachelor of Science in Computer Science,} GPA 8.0/10.0, and 9.58/10.0 (last 60 hours) \\
  Advisors: Prof. Dr. Dieter Rombach and Dipl-Inf. Thorsten Keuler
  \begin{itemize}  \itemsep -2pt
  \item Thesis: Metamodel based Architecture Evaluation of Software Systems
  \end{itemize}

  %\textbf{Technische Universitaet Kaiserslautern} \\ %\hfill 2006--2007 \\
  %{\sl Exchange Program, Computer Science Department}
  %\begin{itemize}  \itemsep -2pt
  %\item  Undergraduate research assistant at Fraunhofer IESE
  %\end{itemize}

%\section{TECHNICAL \\ SKILLS} 5+ years of experience in building distributed application backends for security, finance, and web application domains, with 2 years of storage infrastructure performance analysis and monitoring at Google.
%
%Highly proficient in \textbf{multithreading} and \textbf{C++}.  Proficient in \textbf{Java}, \textbf{Python}, \textbf{MapReduce}, \textbf{JavaScript}, and \textbf{R}.  Beginner in \textbf{Go}.

\section{AWARDS}

\begin{enumerate}
  \item \textbf{Microsoft Research PhD Fellowship}, Class of 2019.
  \item \textbf{EECS Merrill Lynch Graduate Fellowship}, MIT, 2016.
  %\item \textbf{ODGE Tuition Award}, MIT, 2016.
  \item \textbf{Sloan Scholar, Alfred P. Sloan Foundation's MPHD Program}, MIT, 2016.
  \item \textbf{10 Google Peer Bonuses, 6 Google Kudos Awards, 1 Google Spot Bonus}, for technical and professional service contributions.  Spot bonus awarded for internal launch of fleet-wide read / write RPC real-time latency analysis of Colossus clients and related storage servers, 2012--2015.
  \item \textbf{Siebel Scholar, Class of 2011}, awarded for academic excellence and demonstrated leadership to top 5 first-year graduate students from the top 7 CS departments in the world.
  %\item \textbf{Overachievement bonus at the end of internship}, SAP Research - Security \& Trust, France, 2008.
  %\textbf{Best Research Work (``Destaque de Sessao") from the session ``Computer Networks and Fault Tolerance,"} XVII Seminary of Scientific Initiation, Universidade Federal do Rio Grande do Sul, Brazil, October 2005.
  %\textbf{Young Researcher Award (``Jovem Pesquisador") Finalist, ``Earth and Exact Sciences" sessions,} XVII Seminary of Scientific Initiation, October 2005.
\end{enumerate}

\section{TEACHING \& MENTORSHIP}
\begin{enumerate}
  \item Teaching Assistant for {\sl \href{http://dsg.csail.mit.edu/6.S080/}{Software Systems for Data Science (6.080)}(\href{https://github.com/mitdbg/datascienceclass}{github})}, Fall 2019
  \item Mentoring: Mengyuan Sun, Master of Engineering, Fall 2019 and Spring 2020
\end{enumerate}

\section{EXPERIENCE}

  \textbf{Intel}, Portland, OR (remotely from Boston) \hfill Summer 2021 \\
  {\sl Graduate Intern, Intel Optane Group Systems Pathfinding}
  \begin{itemize}
  \item Worked with Dr. Sanjeev Trika and Dr. Jawad Khan on using Optane PMEM for temporal graph analytics. Co-authored patent with Intel collaborators.
  \end{itemize}

  \textbf{Microsoft Research}, New York, NY (remotely from Boston) \hfill Summer 2020 \\
  {\sl Research Intern, AI for Systems Group at MSR NYC (Mentor: Dr. Sid Sen)}
  \begin{itemize}
  \item Evaluated potential benefits of using hybrid KV-store indexing strategy for Azure Redis (internal customers).
  %\item Extended existing hybrid KV-store indexing approach to incorporate RadixSpline, an open source state-of-the-art learned index by folks at MIT DSAIL.
  \end{itemize}

  \textbf{Microsoft}, Redmond, WA \hfill Summer 2017 \\
  {\sl Research Intern, Cloud and Information Services Lab (Mentor. Dr. Carlo Curino)}
  \begin{itemize}
  \item Query optimization for large-scale provenance graphs. Work published at ICDE 2020; co-authored patent with Microsoft collaborators.
  \end{itemize}

  \textbf{Google Inc}, New York, NY and Mountain View, CA (2016) \hfill 2012--2016 \\
  {\sl Software Engineer, Apps} and {\sl Storage Infrastructure}
  \begin{itemize}  \itemsep -2pt
  \item  2015--2016: In charge of infrastructure and monitoring tasks on both backend and frontend components for Google Jamboard.
  \item  2013--2015: Storage Analytics team. Helped local Storage teams with performance-engineering related analysis (e.g., distributed caching and placement policies, RPC latency distributions for read and write OPs). Storage systems included Bigtable, Colossus, Blobstore, and Spanner.
  %\item 2012--2013: Integration of Gmail, Photos and Drive storage metadata (featured on
  %\href{http://goo.gl/y8cpvZ}{TechCrunch}, \href{http://googledrive.blogspot.com/2013/05/bringing-it-all-together-15-gb-now.html}{Google Drive Blog}
  %and \href{https://www.google.com/search?q=google\%20unifies\%20drive\%20gmail\%20google\%20photos\%2015gb}{many others}).
  %\item Techs: C++, MapReduce, Java, Python, JavaScript, R.
  \end{itemize}

%  \textbf{Bloomberg LP}, New York, NY \hfill 2011--2012 \\
%  {\sl Financial Software Developer, Real-time Data Feeds}
%  \begin{itemize}  \itemsep -2pt
%  \item Part of group that is in charge of ingesting and normalizing all real-time data that stock exchanges send to Bloomberg.  Developed and enhanced a number of real-time low-latency market data feed handlers for North and South American exchanges, including Toronto Stock Exchange and Cantor Fitzgerald.
%  \item Primary and/or secondary on-call for 20+ real-time data feeds.
%  \item Basic financial knowledge in fixed income, commodities, equities and derivatives (options and futures) asset classes.
%  \item Techs: multithreading, distributed systems, C++, FIX/FIXML.
%  \end{itemize}

%  \textbf{University of Illinois at Urbana-Champaign} \hfill 2009--2011 \\
%  {\sl Student and Siebel Fellow}, August 2010--June 2011
%  \begin{itemize}  \itemsep -2pt
%  \item Proposed a data partitioning technique for large-scale social network distributed data, co-authoring two papers on it.  This work also served as basis for 2 other MS thesis advised by Prof. Yi Lu from ECE department.
%  \item Wrote MS thesis on graph data query provenance, based on work developed during internship at IBM T. J. Watson Research Center. Advised by Dr. Anastasios Kementsietsidis (IBM) and Prof. Marianne Winslet (Database and Information Systems Lab, UIUC).
%  \item Supported by a Siebel Scholar Fellowship, awarded to top 5 students in the top 5 CS universities in the US.
%  \end{itemize}
%
%  {\sl Research Assistant}, August 2009--May 2010
%  \begin{itemize}  \itemsep -2pt
%  \item Studied techniques towards secure storage and regulatory compliance at Database and Information Systems Lab and DEPEND research groups.
%  \item Co-designed a FPGA-based trusted timestamping platform.
%  \item Co-designed and implemented a microkernel based Android rootkit, presented at the poster session of Oakland 2010.
%  \end{itemize}
%  {\sl Visiting Scholar}, DEPEND research group, January 2009--June 2010
%  \begin{itemize}  \itemsep -2pt
%  \item Performed empirical reliability analysis of virtualized systems through fuzzing of VM hypervisor (VMware ESXi and Xen) interfaces.
%  \item Implemented a fault injection tool that uses VM introspection to corrupt virtual memory addresses and process data structures of Xen Virtual Machines.
%  \item Keywords: C/C++, Xen, VMware ESXi, Fuzzing, Linux Kernel Module Programming
%  \end{itemize}
%
%  \textbf{IBM Research T. J. Watson}, Hawthorne, NY \hfill Summer 2010 \\
%  {\sl Research Intern}
%  \begin{itemize}  \itemsep -2pt
%  \item Interned at the Unified Data Analytics group under Dr. Kavitha Srinivas, and mentored by Dr. Anastasios Kementsietsidis on provenance for large-scale heterogeneous systems.
%  \item Designed extensions to RDF data model and SPARQL query language to support provenance annotations over graph structured data.
%  \item Keywords: data provenance, RDF, SPARQL, graphs, query optimization
%  \end{itemize}

%  \textbf{Google Inc}, Porto Alegre, Brazil \hfill Summer 2008\\
%  {\sl Student Developer, Summer of Code 2008 with Globus Alliance and NCSA}
%  \begin{itemize}  \itemsep -2pt
%  \item Selected for Google Summer of Code Program 2008 with
%  \href{http://www.globus.org}{Globus Alliance}. Designed and implemented
%  \href{https://web.archive.org/web/20150405141413/https://dev.globus.org/wiki/GSoC08/SAML_Holder_of_Key_Authn_for_HTTP_SSO}{SAML Holder-of-key Authentication for Single Sign-On}
%  in \href{http://gridshib.globus.org/}{GridShib}.
%  \item Contributed to an OASIS specification on SAML Holder-of-Key Subject Confirmation submitted to the SSTC in August 2008, and written by Thomas R. Scavo.
%  \item Mentioned twice as a success story at Google Open Source Blog (\url{http://bit.ly/7bpDyu}, \url{http://bit.ly/4xWd6o}).
%  \item Keywords: Java, Identity Management, Single Sign-On, SAML, Shibboleth2, Maven, Ant
%  \end{itemize}
%
%  \textbf{SAP Research}, Mougins, France \hfill November 2007--March 2008 \\
%  {\sl Research Intern, SAP Research in Security \& Trust}
%  \begin{itemize}  \itemsep -2pt
%  \item Participated in the SERENITY E.U. funded project (System Engineering for Security and Dependability) at SAP Research in Security \& Trust.
%  \item Designed and implemented a ``security patterns" library and brokered authentication for Web Services and Workflow SERENITY research prototypes.
%  \item Designed and implemented an API providing support for XML-Encryption and XML-Signature of SOAP messages in SERENITY's workflow applications.
%  \item Filed two patents as outcome of this work (see patents section).
%  \item Keywords: Java, SSO, SAML, SOA, Web Services, WS-Security, Apache Axis2, Apache Rampart.
%  \end{itemize}
%
%  \textbf{Fraunhofer IESE}, Kaiserslautern, Germany \hfill December 2006--October 2007\\
%  {\sl Undergraduate Research Assistant, Product Line Architectures}
%  \begin{itemize}  \itemsep -2pt
%  \item Designed and implemented (i) a domain-specific language to describe architectural metrics and rules, (ii) an algorithm to extract and translate architectural facts to a knowledge base representation that can be interpreted by a Prolog engine, and (iii) an Eclipse based tool to visually aid a software architect during the process of defining architectural metrics, and to perform quantitative assessments of software architectures.
%  \item Wrote my B.S. thesis based on this work (see education section).
%  \item Keywords: Java, Prolog, Model-Driven Architecture, Software Architecture Metrics, Eclipse Plug-in Development, Eclipse Modeling Framework.
%  \end{itemize}
%
%  \textbf{Universidade Federal do Rio Grande do Sul} \hfill 2003--2006 \\
%  {\sl Research Assistant}, Parallel and Distributed Processing group, March 2006--June 2006
%  \begin{itemize}  \itemsep -2pt
%  \item Participated in the design, specification and implementation of a peer-to-peer based
%  network layer, intended for support of Distributed and Massively Multiplayer Games.
%  \item Keywords: C/C++, Network Programming, P2P, Distributed and Massively Multiplayer
%  Games.
%  \end{itemize}
%  {\sl Research Assistant}, Fault Tolerance Research Group, March 2004--Februrary 2006
%  \begin{itemize}  \itemsep -2pt
%  \item Participated in the Dependable Grids project, funded by Hewlett-Packard R\&D Brazil.
%  \item Performed dependability validation of distributed Java applications using communication
%  fault injectors developed by the group.
%  \item Implemented a Java tool that performs off-line synchronization of logs generated by
%  distributed applications.
%  \item Keywords: Java, Fault Injection, Distributed Systems, Distributed Logging, Linux.
%  \end{itemize}

\section{SELECTED PATENTS}
\begin{enumerate}
  \item \textbf{Joana Matos Fonseca da Trindade} (Intel), Jawad Khan (Intel), and Sanjeev Trika (Google), US 20230027351, \href{https://patents.justia.com/patent/20230027351}{``Temporal Graph Analytics on Persistent Memory."} Filed September 21st, 2022.
  \item \textbf{Joana Matos Fonseca da Trindade} (Microsoft), Konstantinos Karanasos (Microsoft), and Carlo Aldo Curino (Microsoft), US US20200265049A1, \href{https://patents.google.com/patent/US20200265049A1/en}{``Materialized graph views for efficient graph analysis."} Filed February 15th, 2019.
  \item \textbf{Joana M. Fonseca da Trindade} (IBM Research T. J. Watson), Anastasios Kementsietsidis (IBM Research T. J. Watson) and Mudhakar Srivatsa (IBM Research T. J. Watson), US 20120327087, \href{http://www.faqs.org/patents/app/20120327087}{``Supporting Recursive Dynamic Provenance Annotations Over Data Graphs."} Filed June 27, 2011.
%A. Benameur (SAP Labs France), \textbf{J. Da Trindade} and P. El-Khoury (SAP Labs France), US 20100162406, \href{http://www.faqs.org/patents/app/20100162406}{``Security Aspects of SOA."}  Filed June 12, 2009.
%A. Benameur (SAP Labs France), \textbf{J. Da Trindade} and P. El-Khoury (SAP Labs France), Europe EP2133831A1, \href{http://www.freepatentsonline.com/EP2133831B1.html}{``Security Aspects of SOA."}  Filed June 12, 2008.
\end{enumerate}

\section{INVITED TALKS}
\begin{enumerate}
  \item ``Kaskade: Graph Views for Efficient Graph Analytics'', University of Chicago (hosted by \href{https://data.cs.uchicago.edu/}{ChiData Group}), May 2020.
  \item ``Kaskade: Graph Views for Efficient Graph Analytics'', ICDE 2020, April 2020.
  \item ``Kaskade: Graph Views for Efficient Graph Analytics'', Microsoft (hosted by \href{https://azuredata.microsoft.com}{Gray Systems Lab}), April 2020.
  \item \href{http://lsds.doc.ic.ac.uk/content/graph-views-efficient-graph-analytics-collaboration-microsoft-cisl}{``Graph Views for Efficient Graph Analytics''}, Imperial College (hosted by \href{https://lsds.doi.ic.ac.uk}{LSDS Group}), April 2018.
\end{enumerate}

\section{SELECTED PUBLICATIONS}
\begin{enumerate}
  %\item \textbf{J. M. F. da Trindade}, J. Shun, S. Madden and N. Tatbul, ``Efficient Temporal Graph Analytics on a Single Machine.'' \textit{Under submission.}
  \item \textbf{J. M. F. da Trindade}, J. Shun, S. Madden and N. Tatbul, ``Kairos: Efficient Temporal Graph Analytics on a Single Machine.'' \textit{NEDB 2023, Cambridge, MA (poster)}
  %\item M. Sun, \textbf{J. M. F. da Trindade}, S. Madden, J. Shun and N. Tatbul, ``In-memory Graph Partitioning for Efficient Temporal Graph Analytics on NVRAM.''  \textit{NEDB 2020, Cambridge, MA (poster)}
  \item \textbf{J. M. F. da Trindade}, K. Karanasos, C. Curino, S. Madden and J. Shun, ``Kaskade: Graph Views for Efficient Graph Analytics.'' \textit{ICDE 2020, Dallas, TX April 2020.}
  \item \textbf{J. M. F. da Trindade}, K. Karanasos, C. Curino, S. Madden and J. Shun, \href{https://arxiv.org/abs/1906.05162}{``Kaskade: Graph Views for Efficient Graph Analytics.''} (arXiv 2019 extended pre-print).
  \item M. Vartak, \textbf{J. M. F. da Trindade}, M. Zaharia and S. Madden, \href{https://cs.stanford.edu/~matei/papers/2018/sigmod_mistique.pdf}{``MISTIQUE: A System to Store and Query Model Intermediates for Model Diagnosis.''} \textit{SIGMOD 2018, Houston, TX, June 2018.}
  \item A. Ilyas, \textbf{J. M. F. da Trindade}, R. C. Fernandez and S. Madden, \href{https://arxiv.org/pdf/1710.11528.pdf}{``Extracting Syntactic Patterns from Databases.''} \textit{ICDE 2018, Paris, France, April 2018.}
%  \item M. Yuan, D. Stein, B. Carrasco, \textbf{J. M. F. da Trindade} and Y. Lu, \href{http://joanatrindade.wdfiles.com/local--files/curriculum/gdm2012-paper.pdf}{``Partitioning Social Networks for Fast Retrieval of Time-dependent Queries.''} \textit{3rd International Workshop on Graph Data Management (GDM, co-located with ICDE), Washington, DC, April 2012.} \textbf{Invited paper}.
%  \item B. Carrasco, Y. Lu and \textbf{J. M. F. da Trindade,} \href{http://research.microsoft.com/en-us/projects/ldg/a04-carrasco.pdf}{``Partitioning Social Networks for Time-dependent Queries."} \textit{4th Workshop on Social Network Systems (SNS, co-located with EuroSys), Salzburg, Austria, April 2011.}
%  \item \textbf{J. M. F. da Trindade,} C. Pham and N. Dautenhahn, \href{http://joanatrindade.wikidot.com/local--files/curriculum/oakland2010-paper.pdf}{``$\mu$BeR: A Microkernel Based Rootkit for Android Smartphones."} \textit{IEEE Symposium on Security and Privacy, Oakland, CA, May 2010} (\href{http://joanatrindade.wikidot.com/local--files/curriculum/oakland2010-paper.pdf}{paper}) (\href{http://joanatrindade.wikidot.com/local--files/curriculum/oakland2010-poster.pdf}{poster}).
\end{enumerate}

\section{SERVICE}
  \textbf{Program Committee}
  \begin{itemize}  \itemsep -2pt
  \item ACM Symposium on Cloud Computing (SoCC) 2021-2023
  \item NeurIPS Temporal Graph Learning Workshop (TGL) 2023
  \end{itemize}

%  \textbf{External Reviewer}
%  \begin{itemize}  \itemsep -2pt
%  \item 11th International Symposium on Stabilization, Safety, and Security of Distributed Systems (SSS 2009)
%  \item 15th IEEE Pacific Rim International Symposium on Dependable Computing (PRDC 2009)
%  \item 2nd IEEE International Conference on Computer Science and its Applications (CSA 2009)
%  \item 25th ACM Symposium On Applied Computing (SAC 2010)
%  \item 40th Annual IEEE/IFIP International Conference on Dependable Systems and Networks (DSN 2010)
%  \item 4th IFIP WG 11.11 International Conference on Trust Management (IFIPTM 2010)
%  \item 7th IEEE International Conference on Autonomic Computing (ICAC 2010)
%  \end{itemize}
%
%  \textbf{Misc Professional Service}
%  \begin{itemize}  \itemsep -2pt
%  \item IEEE Cipher Newsletter reviewer for the technical sessions of the IEEE Symposium on Security and Privacy 2010 (\href{http://www.ieee-security.org/Cipher/PastIssues/2010/E96.May-2010/E96.May-2010.html}{Issue E96, May 31st 2010}).
%  \item Student organizer for ACM Reflections 2009 and Middleware 2009.
%  \end{itemize}

  \textbf{Outreach}
  \begin{itemize}  \itemsep -2pt
  \item Panelist for MIT Grad Student Orientation and Virtual Visit Days diversity recruiting events (2020-2022).
  %\item Tour guide for on-campus graduate housing visits during MIT's open visit days in March 2018.
  \item On-site recruiter with Google Inc. at Grace Hopper Conference 2015. On-campus recruiter with at UIUC with Google Inc. (2013-2015) and Bloomberg L.P. (2012).
  %\item Habitat for Humanity project with Best of Bloomberg (BBOB) philanthropy program.
  %\item Volunteer at Algorithms session at the Go Girls TechKnow 2010, IBM T. J. Watson Research Center.
 % \item Panelist at Google: Robotics tech talk for Girls Who Code, and NYU's CSAW CyberSecurity Program for Young Women 2014 and 2015.
 % \item Technovation Challenge 2014: Mentor of students from Brooklyn International High School that designed and implemented a clothing donation app.
 % \item Per Scholas program: Teaching assistant for a class led by Raymond Blum (Google) on robotics using Arduino.
  \end{itemize}

%\section{OTHER \\ ACTIVITIES}
%  \begin{itemize}  \itemsep -2pt
%  \item Member of EECS Graduate Student Group for MIT Visiting Committee (2022).
%  \item CSAIL Postdoc and Graduate Student Committee (CPGSC): PhD student member representing Systems CoR (2020-2021).
%  \item Radio show host at MIT's WMBR student radio (2016-2018).
%  \item Interests: bass guitar, fantasy and horror comic books, clickety clack mech keebs.
%  \end{itemize}

%\section{LANGUAGES}
%Portuguese (native), English, Spanish (basic)

\end{resume}
\end{document}
